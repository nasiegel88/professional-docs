% !TeX program = XeLaTeX
\documentclass{nihbiosketch}

%\usepackage{draftwatermark}  % delete this in your document!
%\SetWatermarkText{Sample}    % delete this in your document!
%\SetWatermarkLightness{0.9}  % delete this in your document!

%------------------------------------------------------------------------------

\name{Hunt, Morgan Casey}
\eracommons{huntmc}
\position{Associate Professor of Psychology}

\begin{document}
%------------------------------------------------------------------------------

\begin{education}
University of California, Berkeley  & B.S           & 05/1990  & Psychology \\
University of Vermont               & Ph.D.         & 05/1996  & Experimental Psychology \\
University of California, Berkeley  & Postdoctoral  & 08/1998  & Public Health and Epidemiology \\
\end{education}


\section{Personal Statement}

I have the expertise, leadership, training, expertise and motivation necessary
to successfully carry out the proposed research project.  I have a broad
background in psychology, with specific training and expertise in ethnographic
and survey research and secondary data analysis on psychological aspects of
drug addiction.  My research includes neuropsychological changes associated
with addiction.  As PI or co-Investigator on several university- and NIH-
funded grants, I laid the groundwork for the proposed research by developing
effective measures of disability, depression, and other psychosocial factors
relevant to the aging substance abuser, and by establishing strong ties with
community providers that will make it possible to recruit and track
participants over time as documented in the following publications.  In
addition, I successfully administered the projects (e.g.\ staffing, research
protections, budget), collaborated with other researchers, and produced several
peer-reviewed publications from each project.  As a result of these previous
experiences, I am aware of the importance of frequent communication among
project members and of constructing a realistic research plan, timeline, and
budget.  The current application builds logically on my prior work. During
2005--2006 my career was disrupted due to family obligations. However, upon
returning to the field I immediately resumed my research projects and
collaborations and successfully competed for NIH support.

\begin{enumerate}

\item Merryle, R.J. \& Hunt, M.C. (2004). Independent living, physical
        disability and substance abuse among the elderly. Psychology and Aging,
        23(4), 10--22.

\item Hunt, M.C., Jensen, J.L. \& Crenshaw, W. (2007). Substance abuse and
        mental health among community-dwelling elderly. International Journal
        of Geriatric Psychiatry, 24(9), 1124--1135.

\item Hunt, M.C., Wiechelt, S.A. \& Merryle, R. (2008). Predicting the
        substance-abuse treatment needs of an aging population.  American
        Journal of Public Health, 45(2), 236--245. PMCID: PMC9162292 
        
\item Hunt, M.C., Newlin, D.B. \& Fishbein, D. (2009). Brain imaging in
        methamphetamine abusers across the life-span. Gerontology, 46(3),
        122--145.

\end{enumerate}

%------------------------------------------------------------------------------
\section{Positions and Honors}

\subsection*{Positions and Employment}
\begin{datetbl}
1998--2000  & Fellow, Division of Intramural Research, National Institute of Drug Abuse, Bethesda, MD \\
2000--2002  & Lecturer, Department of Psychology, Middlebury College, Middlebury, VT \\
2001--      & Consultant, Coastal Psychological Services, San Francisco, CA  \\
2002--2005  & Assistant Professor, Department of Psychology, Washington University, St.\ Louis, MO \\
2007--      & Associate Professor, Department of Psychology, Washington University, St.\ Louis, MO\\
\end{datetbl}

\subsection*{Other Experience and Professional Memberships}
\begin{datetbl}
1995--           & Member, American Psychological Association \\
1998--           & Member, Gerontological Society of America \\
1998--           & Member, American Geriatrics Society \\
2000--           & Associate Editor, Psychology and Aging \\ 
2003--           & Board of Advisors, Senior Services of Eastern Missouri \\
2003--05         & NIH Peer Review Committee: Psychobiology of Aging, ad hoc reviewer \\
2007--11         & NIH Risk, Adult Addictions Study Section, members \\
\end{datetbl}

\subsection*{Honors}
\begin{datetbl}
2003            & Outstanding Young Faculty Award, Washington University, St.\ Louis, MO \\
2004            & Excellence in Teaching, Washington University, St.\ Louis, MO \\
2009            & Award for Best in Interdisciplinary Ethnography, International Ethnographic Society \\
\end{datetbl}

%------------------------------------------------------------------------------

\section{Contribution to Science}

\begin{enumerate}

\item My early publications directly addressed the fact that substance abuse is
    often overlooked in older adults. However, because many older adults were
    raised during an era of increased drug and alcohol use, there are reasons
    to believe that this will become an increasing issue as the population
    ages.   These publications found that older adults appear in a variety of
    primary care settings or seek mental health providers to deal with emerging
    addiction problems.  These publications document this emerging problem but
    guide primary care providers and geriatric mental health providers to
    recognize symptoms, assess the nature of the problem and apply the
    necessary interventions.   By providing evidence and simple clinical
    approaches, this body of work has changed the standards of care for
    addicted older adults and will continue to provide assistance in relevant
    medical settings well into the future.  I served as the primary
    investigator or co-investigator in all of these studies. 

\begin{enumerate}

\item Gryczynski, J., Shaft, B.M., Merryle, R., \& Hunt, M.C. (2002). Community
        based participatory research with late-life addicts. American Journal
        of Alcohol and Drug Abuse, 15(3), 222--238.

\item Shaft, B.M., Hunt, M.C., Merryle, R., \& Venturi, R. (2003). Policy
        implications of genetic transmission of alcohol and drug abuse in
        female nonusers. International Journal of Drug Policy, 30(5), 46--58.

\item Hunt, M.C., Marks, A.E., Shaft, B.M., Merryle, R., \& Jensen, J.L.
        (2004). Early-life family and community characteristics and late-life
        substance abuse. Journal of Applied Gerontology, 28(2),26--37.

\item Hunt, M.C., Marks, A.E., Venturi, R., Crenshaw, W. \& Ratonian, A.
        (2007). Community-based intervention strategies for reducing alcohol
        and drug abuse in the elderly.  Addiction, 104(9), 1436--1606. PMCID:
        PMC9000292

\end{enumerate}


\item In addition to the contributions described above, with a team of
    collaborators, I directly documented the effectiveness of various
    intervention models for older substance abusers and demonstrated the
    importance of social support networks.   These studies emphasized
    contextual factors in the etiology and maintenance of addictive disorders
    and the disruptive potential of networks in substance abuse treatment. This
    body of work also discusses the prevalence of alcohol, amphetamine, and
    opioid abuse in older adults and how networking approaches can be used to
    mitigate the effects of these disorders.    

\begin{enumerate}

\item Hunt, M.C., Merryle, R. \& Jensen, J.L. (2005). The effect of social
        support networks on morbidity among elderly substance abusers. Journal
        of the American Geriatrics Society, 57(4), 15--23.

\item Hunt, M.C., Pour, B., Marks, A.E., Merryle, R. \& Jensen, J.L. (2005).
        Aging out of methadone treatment. American Journal of Alcohol and Drug
        Abuse, 15(6), 134--149. 

\item Merryle, R. \& Hunt, M.C. (2007). Randomized clinical trial of cotinine
        in older nicotine addicts. Age and Ageing, 38(2), 9--23. PMCID:
        PMC9002364

\end{enumerate}

\item Methadone maintenance has been used to treat narcotics addicts for many
    years but I led research that  has shown that over the long-term, those in
    methadone treatment view themselves negatively and they gradually begin to
    view treatment as an intrusion into normal life.   Elderly narcotics users
    were shown in carefully constructed ethnographic studies to be especially
    responsive to tailored social support networks that allow them to
    eventually reduce their maintenance doses and move into other forms of
    therapy.  These studies also demonstrate the policy and commercial
    implications associated with these findings.

\begin{enumerate}   

\item Hunt, M.C. \& Jensen, J.L. (2003). Morbidity among elderly substance
        abusers. Journal of the Geriatrics, 60(4), 45--61.

\item Hunt, M.C. \& Pour, B. (2004). Methadone treatment and personal
        assessment. Journal Drug Abuse, 45(5), 15--26. 

\item  Merryle, R. \& Hunt, M.C. (2005). The use of various nicotine delivery
        systems by older nicotine addicts. Journal of Ageing, 54(1), 24--41.
        PMCID: PMC9112304

\item Hunt, M.C., Jensen, J.L. \& Merryle, R. (2008). The aging addict:
        ethnographic profiles of the elderly drug user.  NY, NY: W. W. Norton
        \& Company.

\end{enumerate}

\end{enumerate}

\subsection*{Complete List of Published Work in MyBibliography:} 
\url{http://www.ncbi.nlm.nih.gov/myncbi/browse/collection/45972964/}


%------------------------------------------------------------------------------

\section{Research Support}

\subsection*{Ongoing Research Support}

\grantinfo{R01 DA942367}{Hunt (PI)}{09/01/08--08/31/16}
{Health trajectories and behavioral interventions among older substance abusers}
{The goal of this study is to compare the effects of two substance abuse interventions on health 
outcomes in an urban population of older opiate addicts.}
{Role: PI}

\bigskip

\grantinfo{R01 MH922731}{Merryle (PI)}{12/15/07--11/30/15}
{Physical disability, depression and substance abuse in the elderly}
{The goal of this study is to identify disability and depression trajectories and demographic factors 
associated with substance abuse in an independently-living elderly population.}
{Role: Co-Investigator}

\bigskip

\grantinfo{Faculty Resources Grant, Washington University}{08/15/09--08/14/15}
{Opiate Addiction Database}
{The goal of this project is to create an integrated database of demographic, social and biomedical 
information for homeless opiate abusers in two urban Missouri locations, using a number of state and 
local data sources.}
{Role: PI}


%------------------------------------------------------------------------------

\subsection*{Completed Research Support}

\grantinfo{R21 AA998075}{Hunt (PI)}{01/01/11--12/31/13}
{Community-based intervention for alcohol abuse}
{The goal of this project was to assess a community-based strategy for reducing alcohol abuse among 
older individuals.}
{Role: PI}



\end{document}

