% !TeX program = XeLaTeX
\documentclass{nihbiosketch}

%\usepackage{draftwatermark}  % delete this in your document!
%\SetWatermarkText{Sample}    % delete this in your document!
%\SetWatermarkLightness{0.9}  % delete this in your document!

%------------------------------------------------------------------------------

\name{Siegel, Noah Alexander}
\eracommons{NOAHSIEGEL}
\position{Graduate Student Reseacher}

\begin{document}
%------------------------------------------------------------------------------

\begin{education}
California State University, East Bay  & B.S           & 06/2018  & Biology \\
University of California, Davis               & Ph.D.         & 06/2023  & Immmunology \\
%University of California, Berkeley  & Postdoctoral  & 08/1998  & Public Health and Epidemiology \\
\end{education}


\section{Personal Statement}

I am a first-year graduate student researcher in the lab of Dr. Lisa Miller at the Respiratory Disease Unit located in the California National Primate Center. 
Much of my laboratory preparation comes from the time I spent as an undergraduate in the lab Dr. Chul Kim at the California State University, East Bay. There, I became acquainted classical biophysical and biochemical techniques such as FPLC and Fluorospectroscopy. With my unique skills, I hope to provide novel insights in addressing research questions. Furthermore, I plan to utilize common mathematical and statistical software, such a R studio, to investigate differentially expressed genes in rhesus macaques in the context of various lung related insults.
For this application, I bring exceptional skills in data analysis and, with my prior experience working under a physics instructor, mathematics. To the best of my knowledge, there are no other predoctoral candidates who offer such a unique perspective and training. 
% \begin{enumerate}

% \item Merryle, R.J. \& Hunt, M.C. (2004). Independent living, physical
%         disability and substance abuse among the elderly. Psychology and Aging,
%         23(4), 10--22.

% \item Hunt, M.C., Jensen, J.L. \& Crenshaw, W. (2007). Substance abuse and
%         mental health among community-dwelling elderly. International Journal
%         of Geriatric Psychiatry, 24(9), 1124--1135.

% \item Hunt, M.C., Wiechelt, S.A. \& Merryle, R. (2008). Predicting the
%         substance-abuse treatment needs of an aging population.  American
%         Journal of Public Health, 45(2), 236--245. PMCID: PMC9162292 
        
% \item Hunt, M.C., Newlin, D.B. \& Fishbein, D. (2009). Brain imaging in
%         methamphetamine abusers across the life-span. Gerontology, 46(3),
%         122--145.

% \end{enumerate}

%------------------------------------------------------------------------------

\subsection*{Positions and Employment}
\begin{datetbl}
2016--2018 & Supplemental Instructional Leader, Student Center for Academic Achievement, California State University, East Bay \\
2017--2018  & Undergraduate Student Researcher, Kim Lab, California State University, East Bay \\
2018--      & Graduate Student Researcher, Respiratory Disease Unit at the CNPRC, University of California, Davis  \\
\end{datetbl}

\subsection*{Other Experience and Professional Memberships}
\begin{datetbl}
2019--           & Member, American Thoracic Society \\
\end{datetbl}

\subsection*{Honors}
\begin{datetbl}
2019            & California National Primate Research Center Diversity Fellowship, University of California, Davis, Davis, CA \\
% 2004            & Excellence in Teaching, Washington University, St.\ Louis, MO \\
% 2009            & Award for Best in Interdisciplinary Ethnography, International Ethnographic Society \\
\end{datetbl}

%------------------------------------------------------------------------------

\section{Contribution to Science}

\begin{enumerate}

\item As an undergraduate student, I volunteered for over a year in a structural biochemistry lab at the California State University, East Bay. There, the lab and I worked on a binding study to investigate critical RNA-protein interactions of the Brome Mosaic Virus. My primary role on the project was sample preparation and analysis for quenching and anisotropy experiments. We also utilized size exclusion chromatography via FPLC to determine the binding strength between mutant variants of a known pseudoknot of the BMV and the virus capsid protein. In these experiments, I prepared samples and measured fractions of various concentrations of RNA. Furthermore, I aided the masters students in the lab with interpreting and analyzed collected data. We presented our findings at a California Student Research poster event in April 2018.

% \begin{enumerate}

% \item Gryczynski, J., Shaft, B.M., Merryle, R., \& Hunt, M.C. (2002). Community
%         based participatory research with late-life addicts. American Journal
%         of Alcohol and Drug Abuse, 15(3), 222--238.

% \item Shaft, B.M., Hunt, M.C., Merryle, R., \& Venturi, R. (2003). Policy
%         implications of genetic transmission of alcohol and drug abuse in
%         female nonusers. International Journal of Drug Policy, 30(5), 46--58.

% \item Hunt, M.C., Marks, A.E., Shaft, B.M., Merryle, R., \& Jensen, J.L.
%         (2004). Early-life family and community characteristics and late-life
%         substance abuse. Journal of Applied Gerontology, 28(2),26--37.

% \item Hunt, M.C., Marks, A.E., Venturi, R., Crenshaw, W. \& Ratonian, A.
%         (2007). Community-based intervention strategies for reducing alcohol
%         and drug abuse in the elderly.  Addiction, 104(9), 1436--1606. PMCID:
%         PMC9000292

\end{enumerate}


% \item In addition to the contributions described above, with a team of
%     collaborators, I directly documented the effectiveness of various
%     intervention models for older substance abusers and demonstrated the
%     importance of social support networks.   These studies emphasized
%     contextual factors in the etiology and maintenance of addictive disorders
%     and the disruptive potential of networks in substance abuse treatment. This
%     body of work also discusses the prevalence of alcohol, amphetamine, and
%     opioid abuse in older adults and how networking approaches can be used to
%     mitigate the effects of these disorders.    

% \begin{enumerate}

% \item Hunt, M.C., Merryle, R. \& Jensen, J.L. (2005). The effect of social
%         support networks on morbidity among elderly substance abusers. Journal
%         of the American Geriatrics Society, 57(4), 15--23.

% \item Hunt, M.C., Pour, B., Marks, A.E., Merryle, R. \& Jensen, J.L. (2005).
%         Aging out of methadone treatment. American Journal of Alcohol and Drug
%         Abuse, 15(6), 134--149. 

% \item Merryle, R. \& Hunt, M.C. (2007). Randomized clinical trial of cotinine
%         in older nicotine addicts. Age and Ageing, 38(2), 9--23. PMCID:
%         PMC9002364

% \end{enumerate}

% \item Methadone maintenance has been used to treat narcotics addicts for many
%     years but I led research that  has shown that over the long-term, those in
%     methadone treatment view themselves negatively and they gradually begin to
%     view treatment as an intrusion into normal life.   Elderly narcotics users
%     were shown in carefully constructed ethnographic studies to be especially
%     responsive to tailored social support networks that allow them to
%     eventually reduce their maintenance doses and move into other forms of
%     therapy.  These studies also demonstrate the policy and commercial
%     implications associated with these findings.

% \begin{enumerate}   

% \item Hunt, M.C. \& Jensen, J.L. (2003). Morbidity among elderly substance
%         abusers. Journal of the Geriatrics, 60(4), 45--61.

% \item Hunt, M.C. \& Pour, B. (2004). Methadone treatment and personal
%         assessment. Journal Drug Abuse, 45(5), 15--26. 

% \item  Merryle, R. \& Hunt, M.C. (2005). The use of various nicotine delivery
%         systems by older nicotine addicts. Journal of Ageing, 54(1), 24--41.
%         PMCID: PMC9112304

% \item Hunt, M.C., Jensen, J.L. \& Merryle, R. (2008). The aging addict:
%         ethnographic profiles of the elderly drug user.  NY, NY: W. W. Norton
%         \& Company.

% \end{enumerate}

% \end{enumerate}

% \subsection*{Complete List of Published Work in MyBibliography:} 
% \url{http://www.ncbi.nlm.nih.gov/myncbi/browse/collection/45972964/}


%------------------------------------------------------------------------------

% \section{Research Support}

% \subsection*{Ongoing Research Support}

% \grantinfo{R01 DA942367}{Hunt (PI)}{09/01/08--08/31/16}
% {Health trajectories and behavioral interventions among older substance abusers}
% {The goal of this study is to compare the effects of two substance abuse interventions on health 
% outcomes in an urban population of older opiate addicts.}
% {Role: PI}

% \bigskip

% \grantinfo{R01 MH922731}{Merryle (PI)}{12/15/07--11/30/15}
% {Physical disability, depression and substance abuse in the elderly}
% {The goal of this study is to identify disability and depression trajectories and demographic factors 
% associated with substance abuse in an independently-living elderly population.}
% {Role: Co-Investigator}

% \bigskip

% \grantinfo{Faculty Resources Grant, Washington University}{08/15/09--08/14/15}
% {Opiate Addiction Database}
% {The goal of this project is to create an integrated database of demographic, social and biomedical 
% information for homeless opiate abusers in two urban Missouri locations, using a number of state and 
% local data sources.}
% {Role: PI}


% %------------------------------------------------------------------------------

% \subsection*{Completed Research Support}

% \grantinfo{R21 AA998075}{Hunt (PI)}{01/01/11--12/31/13}
% {Community-based intervention for alcohol abuse}
% {The goal of this project was to assess a community-based strategy for reducing alcohol abuse among 
% older individuals.}
% {Role: PI}



\end{document}